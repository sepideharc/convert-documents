\subsection{The Dead End of American Archaeologists in
Iran}\label{the-dead-end-of-american-archaeologists-in-iran}

Despite all the optimism spurred about all the potentials of opening new
windows of opportunity for the Iranian civil society in the aftermath of
the election of a moderate government and followed by the proposed
nuclear agreement to lift sanctions in exchange of putting cap on Iran's
nuclear activities, so far not much has changed. This is especially true
with regard to the status of academic and scientific exchange between
the United States and Iran.

I look at this issue from my own perspective based on my background in
archaeology. There is a long and rather productive history of Iranian
archaeology in the United States. After 1930 when the ridiculous French
monopoly on all archaeological excavations in Iran expired, gates opened
for others. Robert Braidwood from the Oriental Institute of the
University of Chicago was one of the first Americans to excavate in
Iran. His pioneering work in the Zagros region made a breakthrough in
discovering the origins of farming and food production in the ancient
Near East. Later during the 70's, the involvement of the American
Universities in archaeological excavations heightened. Important
archaeological surveys and excavations in the Dehluran plain, sites of
\emph{Hasanloo, Malyan, Tappeh Yahya} trained the next generation of
prominent near eastern archaeologists whose work focused heavily on the
Iranian cultural sphere. Like many other fields, this trend ended after
the Islamic Revolution. There has been some sporadic collaboration among
the Iranian and American archaeologists, but due to their short-term and
unpredictable nature there has not been any significant results. Which
would be namely, the training of a new generation of Near Eastern
archaeologists whose work focuses on Iran.

Before the 1979 Islamic Revolution, the United States and Iran had deep
academic ties and at one time, Iran sent more students to the U.S. than
any other country. Today, although Persian is considered a critical
language in the United States, very few college graduates have
proficiency in it. Most US students interested in Persian are limited to
studying in Tajikistan. Many would like to visit and study in Iran, but
very few have the opportunity to do so. Naturally this situation has
affected the field of Iranian Archaeology in the US significantly.

A few weeks ago I participated in the annual conference of the American
School of Oriental Research \emph{{[}ASOR{]}}(http://www.asor.org/) .
The session engaging the archaeology of Iran was ever more minimized
compared to all the other long-term projects conducted everywhere else
in the Near East (with their considerable depth and breadth). Today, The
Oriental Institute of the University of Chicago is the sole leader of
the Iranian archaeology in the US thanks to their rich archives and
Persepolis tablets (which were recently saved from being sold, thanks to
the appeal court reversing the previous ruling). The University of
Pennsylvania and to a certain degree UCLA are following U. Chicago's
lead. All of this is despite the fact that in recent history the number
of these departments involved in studying the different eras of Iranian
past was many more.

This is especially unfortunate because the American school of
archaeology is different from and better than other schools of
archaeology. I am critical of the constructed idea of American
exceptionalism, but in this case I argue for the uniqueness of the
approach American archaeology takes to study the human past. In America
archeology is one of the branches under the four-field anthropology
founded by Franz Boas. This school is characterized against others (such
as European schools) by conceiving anthropology as the integration of
many different objects into one overarching subject: the human being.
This ``anthropological archaeology `` is beyond the discourses of race
and nationalism, defining its object of inquiry as the human species in
totality. It considers the essence of the human species to be their
tremendous variation in human form and activity and seeks to reveal
human resiliency, ingenuity and creativity through thousands of years
without attaching evolutionary value to any of them. As Ruth Benedict
once said: ``the purpose of {[}this{]} anthropology is to make the world
safe for human differences''. In the face of a growing populist
nationalism in Iran that appears to have concerned at least some
official and religious figures, the universal approach that American
archaeology takes to understand the human past (and ``record'') is a
much needed discourse inside Iran's academia as well as its civil
society.
